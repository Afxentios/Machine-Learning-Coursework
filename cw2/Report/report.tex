\documentclass[a4paper,11pt]{article}
\usepackage[margin=2cm]{geometry}
%\usepackage{anysize}
\usepackage[pdftex]{graphicx}
\usepackage{url}
\usepackage{fixltx2e}
\usepackage{listings}
\usepackage{textcomp}
\usepackage{wrapfig}
\usepackage{color}
\usepackage{subfig}
\usepackage{fancyhdr}
\usepackage{newclude}
\usepackage[nodayofweek]{datetime}
\usepackage[small,compact]{titlesec}
\usepackage[pdfborder=0]{hyperref}
\longdate

\setlength{\parskip}{11pt} 
\setlength\parindent{0pt}

\pagestyle{fancyplain}
\fancyhf{}
\lhead{\fancyplain{}{Machine Learning CBC}}
\rhead{\fancyplain{}{\today}}
\cfoot{\fancyplain{}{\thepage}}


\title{395 Machine Learning\\\Large{--- Assignment 3 ---}}
\author{Group 7\\Porfyrios Vasileiou, Afxentios Hadjiminas, John Flanagan.\\
       \{pv311, ah2411, jf311.\}@doc.ic.ac.uk\\ \\
       \small{CBC helper: Ioannis Marras}\\
       \small{Course: CO395, Imperial College London}
}


\begin{document}
\maketitle

\section{Implementation}
The scope of this coursework is to acquire knowledge about how neural networks work in Matlab and how can they be trained for classifying. Furthermore, to find the best parameters for the network that will output the best results. Parameters such as the training function, learning rule and rate, activation function and more. It is also required to compare the performance of multiple single output networks instead of one single multiple output network. The performance is again computed using 10-Fold Cross-Validation with he same approach as for decision trees. The whole examples table is divided each time into test and train with the test set occupying the 10\% of the total examples. Each fold the actual and predicted labels are found and in the end we calculate the average confusion matrix. Using the matrix we can then calculate the average precision, recall and F1 measure. The networks are created using the function feedforwardnet() where we assign the number of hidden layers and the training function. In the last assignment each row was an example with the attributes assigned to the columns. For neural networks each column is a different example and the attributes lay in the rows. For this transformation we use the given function NNdata(x, y) that is provided. 


\section{Network Parameters}
For deciding the best network for the scope of this coursework experiments were made with various network parameters. For instance networks were tested with different numbers of hidden layers and the number of neurons in each layer. We were also changing the number of learning rate each time. Additionally we trained the network with various transfer functions and training functions in order to find the best combination that outputs better results. This procedure took a lot of time, however it was essential in spotting the best network. These parameters are discussed thoroughly in the next section.

\subsection{Training Function}
The most important parameter of the network is the training function. After testing multiple of these functions it was found that trainscg and trainbr were the ones with the best performance and minimum average error between 4\% and 8\%. A reason for this is that they do a very good job with early stopping and regularisation. However in most cases trainscg had slightly better results and was significantly faster. Other training functions that were tested were trainlm, traingd and traingdm. Trainlm achieved 9\% average error when using 2 hidden layers of 20 neurons each whereas traingd and traingdm both produced relatively high average errors for 1 and 2 hidden layers between 15\% and 40\%.

\subsection{Learning rate}
Learning rate is another important parameter that must be set in order to get good results. Values between 0.01 and 0.1 were used. It was observed that using high rates was producing bad results and same applied for too low rates. In the end, it was decided to use 0.05 as the learning rate because it produced the lowest error and hence better results.

\subsection{Transfer Function}
For deciding the best transfer function we tested tansig, logsig and purelin. These tests were made for both network approaches. The worst results came from using �purelin� as we observed overfitting and early stopping was not efficient. We achieved best results with �tansig� where we received the minimum error of the three functions. �logsig� was the faster with slightly worse results than �tansig�.

\subsection{Epochs}
Since early stopping and regularisation methods were applied, the number of epochs seemed to never reach or pass 100. Therefore it was set to 100 that was more than enough for this coursework.

\subsection{Topology}
For deciding the appropriate Topology of the network we have experimented with different numbers of hidden layers as well a different number of neurons for each layer. After producing the average error for each case we decided to use 1 or 2 hidden layers. For the first case we use 30-50 neurons whereas for 2 hidden layers we assign 30 neurons for each layer. Using more hidden layers produced bad results as the complexity of the network and training time increased. In addition using many hidden layers increases the probability of overfitting and over training. 	

1 hidden layers
(10):0.07
(35):0.05
(45):0.06
(50):0.02
(60):0.04
(100):0.14

2 hidden layers 
(10 10): 0.14
(30,30): 0.06 

3 hidden layers 
(10 10 10):0.20

\section{Regularization Methodology}
The main problem we faced during the training of the Neural Network was the overfitting. The network has been trained with a very low error rate, however when the new data was included, the performance was very bad with a high error rate. That�s because the network memorized the samples from the training set but it was not able to adapt to the new situation, resulting overfitting. In order to solve this problem by improving the generalization, two methods have been adopted.

\subsection{Modified Performance Function}
In this method, the performance function has been changed. The default performance function is the mean sum of squares of the network errors on the training set (MSE). To improve the generalization a term that consists of the mean of the sum of squares of the network weights and biases has been added on the MSE resulting the MSEREG function. Using this performance function (net.performFcn = msereg) the neural network has smaller weights and biases, and as a result the network is less likely to overfit and its outcome is smoother.


\subsection{Bayesian Regularisation}
Another way to improve the generalization though the regularisation for small datasets is the Bayesian Regularisation. That�s why the trainbr function which implements the Bayesian regularization has been tested. However after several tests the team decided that the performance of network using the training function trainscg along with the performance function msereg is better comparing to the network with the trainbr function.

\subsection{Early Stopping}
The second method that has been used to avoid overfitting is to stop the training of the network at an earlier point. For this purpose, Matlab offers modification option on the training function. In the specific neural network the following training parameters of the trainscg has been set up. Firstly, the net.trainingParam.goal(=0.01) has been set up in order to stop the training when the network has achieved classifying correctly 99\% of all examples. Secondly, the net.trainParam.min\_grad which is responsible for stopping training when the gradient of the network performance reaches a certain minimum has been used with the value 0.01. This value was chosen so that training stops when the error rate decreases by a small amount but not smaller than 0.01.  Lastly, the parameters net.trainParam.epochs was chosen to get the value of 100 maximum epochs, based on the experiment results which has been discussed in the previous section.

\section{Validation Results}
This section displays the results of both network types after using the 10-Fold Cross-Validation method. 
\subsection{Performance(F1 Measure) per fold}
These graphs show the mean F1 measure computed for each fold

\subsection{10-Fold Cross-Validation Average Results}
Below are the final average Confusion Matrices, Recall, Precision rates and the F1 measure of both network approaches. 

\section{Single Output Vs Multi Output Networks}
For this assignment we were instructed to approach the problem with two different types of networks. The first one was to create one network that produces six outputs of 0 and 1 each that represent the emotion. For example the output 001000 classifies the example to the third emotion. The other approach was to create a single-output network for every emotion(label) and join the results in the end.

\end{document}
